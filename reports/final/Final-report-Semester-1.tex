\documentclass[12pt, a4paper]{article}
\usepackage{graphicx}
\usepackage{geometry}
\usepackage[utf8]{inputenc}
\graphicspath{{./images}}

\geometry{top=3cm}

\title{KWAME NKRUMAH UNIVERSITY OF SCIENCE AND TECHNOLOGY\\
COLLEGE OF ENGINEERING\\
DEPARTMENT OF COMPUTER ENGINEERING\\ 
\includegraphics[scale=0.4]{knust} 
\\ALLOCATION OF SENSORS TO MONITOR VITAL SOIL AND ATMOSPHERIC FACTORS THAT AFFECT PLANT GROWTH\\ Case Study: KNUST Peasant Farms}
\author{Amoah Junior Albert - 3022620\\ Bonney Samuel Nii Awuley - 3028320 \\ Lawson Isaac Nii Lante - 3033820 \\ \\ PROJECT SUPERVISOR\\ Dipl. -Ing Benjamin Nii Kommey}

\begin{document}
\maketitle
\thispagestyle{empty}
\newpage

\pagenumbering{roman}
\begin{center}
\textbf{DECLARATION}
\end{center}
This is to certify that the project title \textbf{``Allocation Of Sensors To Monitor Vital Soil And Atmospheric Factors That Affect Plant Growth''} submitted to Kwame Nkrumah University of Science and Technology, represents our original work. The project was conducted under the supervision of Ing. Benjamin Kommey, within the Department of Computer Engineering in the College of Engineering as a partial requirement for the degree of BSc. Computer Engineering.\\ \\ \\  \\ \\ \\
Signature....................................... Date ........................................\\ \\
Amoah Albert Junior(Candidate)\\ \\ \\ \\
Signature....................................... Date ........................................\\ \\
Bonney Nii Awuley(Candidate)\\ \\  \\ \\
Signature....................................... Date ........................................\\ \\
Lawson Isaac Nii Lante(Candidate)\\ \\ \\ \\
Signature....................................... Date ........................................\\ \\
Dipl. -Ing Benjamin Nii Kommey(Supervisor)
\newpage

\begin{center}
\textbf{DEDICATION}
\end{center}
\newpage

\begin{center}
\textbf{ACKNOWLEGEMENT}
\end{center}
\newpage

\begin{center}
\textbf{ABSTRACT}
\end{center}

\newpage


\tableofcontents
\newpage
\pagenumbering{arabic}
\section{Introduction}
\subsection{Background of Study}
Farming is a very important aspect of humanity. Humans cannot live without food and food cannot be without farms, hence the neccessity of farms. Farmers do a great service to man. As the SDG goal number goes: \textbf{Zero Hunger}; End hunger, achieve food security and improved nutrition and promote sustainable agriculture. To achieve this we need to take care of agriculture and improve on its productivity. Agricultural productivity is key to eradicating hunger and achieving food security. As stated in the SDG goal, "promote sustainable agriculture" sustainable agriculture is a much needed achievement. Agricultural productivity can be improved by lay down and implementing measures that maximise crop yield. There are various many factors that affect plant growth and therefore their yield. Plants need certain essentials to grow well and these essentials if provided accurately can increase their growth and therefore increase productivity.
\\
Several factors affect crop health and contribute in the growth of plants soil factors, atmospheric factors and pests\\Soil factors include soil temperature, the soil pH, supply of soil nutrients, soil moisture etc. Atmospheric factors also include atmospheric temperature, humidity, precipitation or rain etc. Soil temperature directly affects plant growth. Most soil organisms function best at an optimum soil temperature. Soil temperature impacts the rate of nitrifi- cation. It also influences soil moisture content, aeration and availability of plant nutrients. Soil pH will influence both the availability of soil nutrients to plants and how the nutrients react with each other. For example: At a low pH, many elements become less available to plants, while others such as iron, aluminum and manganese become toxic to plants. Soil moisture is a measure of soil health, the water content present in a certain area of the ground. All plants need to be in a specific soil moisture range — the majority of plants thrive in soil with a moisture level that ranges between 20 and 60 percent. This is important because if you're trying to grow certain plants, you not only have to make sure the soil is fertile enough to support growth, you have to be able to keep soil moisture in a certain range. It's an issue for industrial growers, but also private gardeners and anyone trying to grow their own vegetables. This is because the water content in soil is a solvent, meaning that it breaks down the nutrients and minerals that plants need from the dirt, allowing them to absorb these helpful particles into their systems. As atmospheric temperature increases (up to a point), photosynthesis, transpiration and respiration increase. When combined with day length, temperature also affects the change from vegetative (leafy) to reproductive (flowering) growth. Too much rain during germination may saturate soils, resulting in poor germination and reduced stands. However, too little rain during germination may reduce germination and leave plants ill-prepared for future growth and development challenges. When relative humidity levels are too high or there is a lack of air circulation, a plant cannot make water evaporate (part of the transpiration process) or draw nutrients from the soil. When this occurs for a prolonged period, a plant eventually rots.\\ \\
From the above study we can infer that realtime and accurate information on the condition of a farm in needed.
\subsection{Problem Statement}
Agricultural farms are important to humans. They produce the food we eat. The food we eat are the yields of the crops that are cultivated.\\
Farmers frequently face challenges in providing the necessary farm essentials to support ideal plant growth, primarily due to inadequate and inaccurate farm condition data. The problem is that farmers have been providing farm essentials to crops based on guess work. They are not able to accurately tell the needs of the farms.
\\
Current farm management practices are often reactive and data-deficient, leading to inefficiencies and unsustainable resource use across various aspects of agricultural production Farmers struggle to optimize irrigation, fertilization, and pest control due to limited real-time data on soil health, weather conditions, and crop development, resulting in potential yield losses, water waste, and environmental damage from excess nutrients.
\\
Therefore, there is a critical need for a farm monitoring system that:
\begin{itemize}
 \item Provides real-time, comprehensive data on environmental conditions, crop health.
 \item Offers user-friendly interfaces and actionable insights for informed decision-making.
 \item Is cost-effective, scalable, and accessible to farmers of all sizes.
 \item Contributes to sustainable agriculture practices by optimizing resources
\end{itemize}
By addressing these challenges a well-designed system that can monitor vital soil and atmospheric factors that affect farm plant growth can empower farmers to achieve: increased productivity and profitability, enhanced decision-making and long-term sustainability.



\newpage
\subsection{Objectives of the Project}
\subsubsection{General Objectives}
The project's objective is to develop a streamlined approach for deploying sensors to monitor critical soil and atmospheric conditions essential for plant growth. This involves identifying the optimal quantity, types, and placements of sensors to collect data on variables such as moisture levels, nutrient content, temperature, humidity, and atmospheric gases. Ultimately, the aim is to equip farmers with timely information to refine their farming techniques, resulting in higher crop yields and quality, decreased resource consumption, and a reduced environmental impact.
\subsubsection{Specific Objectives}
This project aims to achieve several key objectives:\\
\begin{enumerate}
\item \textbf{Increased efficiency and productivity:}
\begin{enumerate}
 \item Optimize resource use (water, fertilizer, energy) based on real-time data.
 \item Automate tasks like irrigation and climate control to save time and labor.
 \item Improve yield and crop quality through early detection of problems.
\end{enumerate}
 
\item \textbf{Reduced environmental impact:} Minimize water usage and fertilizer runoff through precise application.\\
\item \textbf{Improved decision-making:} Facilitate informed decision-making for improved farm management and profitability.
\end{enumerate}

\newpage
\subsection{Scope of the project}
The focus of this project is to design a system that can be used to monitor any size of open farms.
\begin{itemize}
\item[--] The system will measure specific soil and atmospheric factors. Atmospheric factors include: atmospheric temperature, humidity, and rain/precipitation. Soil factors also include: soil pH, soil moisture, soil temperature and supply of specific nutrients.
\item[--] The whole system is a Wireless Sensor Network(WSN) based implementation been used to  monitor the farm.
\item[--] The system will also analyze the information gathered via the sensors and then sent to a cloud service for processing. Based on the information derived, farmers will be advised on what to do.
\item[--] The system will have a frontend interface where information derived can be displayed and farmers recieve advise.
\end{itemize}

\newpage
\subsection{Significance of Study}
Allocation of sensors to monitor vital soil and atmospheric factors is very essential. The can help improve on our agricultural prowess and productivity. 
\\ \\
\textbf{Optimization of Resources:} The system can help farmers optimize the use of resources such as water, fertilizers, and pesticides, reducing waste and increasing efficiency.
\\ \\
\textbf{Early Detection of Issues:} This system can detect issues such as pest infestations, disease outbreaks, or nutrient deficiencies early, allowing farmers to take timely action to mitigate losses.
\\ \\
\textbf{Data-Driven Decision Making:} By collecting and analyzing data on crop health, weather patterns, soil moisture, and other factors, farmers can make informed decisions to improve yields and profitability.
\\ \\
\textbf{Remote Monitoring:} With this system, farmers can remotely monitor their fields allowing them to keep an eye on their operations even when they are not physically present.
\\ \\
\textbf{Environmental Sustainability:} By monitoring factors like soil health and water usage, farmers can adopt more sustainable practices that minimize environmental impact and conserve natural resources.
\\ \\
\textbf{Increase in Productivity:} By providing real-time insights into crop conditions, monitoring systems can help farmers increase productivity and ultimately profitability.


Overall, farm monitoring systems enable farmers to make more informed decisions, increase efficiency, and ultimately improve the sustainability and profitability of their operations.

\newpage
\subsection{Organisation of Study}
\begin{enumerate}
\item \textbf{Introduction}
	\begin{itemize}
	\item[--] Aim and objectives of the project.
	\item[--] Background information on how certain soil and atmospheric factors affect plant    growth.
	\end{itemize}
\item \textbf{Farm monitoring systems}
\begin{itemize}
    \item[--] Types of farm monitoring systems implemented.
	\item[--] Benefits and limitations of existing technology.
\end{itemize}
\item \textbf{Design of System}
\begin{itemize}
	\item[--] Design of system architecture.
    \item[--] Physical design consideration for the system.
    \item[--] Components of the smart monitoring system.
\end{itemize}
\item \textbf{Programming of the System}
\begin{itemize}
	\item[--] Appropriate programming laguages that can be used for programming the system and controling the various components.
	\item[--] Algorithims that would be used to gather and analyze information.
\end{itemize}
\item \textbf{Prototype development}
\begin{itemize}
\item[--] Building a prototype of the monitoring system
\item[--] Testing the prototype for efficiency and functionality
\item[--] Iterative improvement of the prototype system
\end{itemize}
\item \textbf{Conclusion}
\begin{itemize}
	\item[--] Summary of the study.
	\item[--] Future scope of the project.
\end{itemize}
\end{enumerate}
\newpage

\section{Literature Review}
\subsection{Introduction}
Over the past few years, the need for precision and accurate agriculture has been on the rise. This is all because it has been found out that resources are wasted or not used adequately during farming. For instance water; farms could easily be under-irrigated or over-irrigated and go on unnoticed, Fertilizers could be overused or underused. This is all because there is no way of accurately telling the condition of the soil the plant is embedded in or the environmental factors that affect it. Precision agriculture eliminates guess work, encourages data-driven decisions and helps mitigate the wastage of resources. Due to these advantages in the field of agriculture, there has been the proliferation of serveral technologies to solve this problem efficiently. In this section of the report, articles on studies and implementations of such technologies are discussed.
\subsection{Related Works}
\subsubsection{Design and Deploy a Wireless Sensor Network for Precision Agriculture}
In an article by Tuan Din Le and Dat Ho Tan from Department of Computer Science Long An University of Economics and Industry in 2015 during the second National Foundation for Science and Technology Development Conference on Information and Computer Science a similar project was discussed. In their approach, a Wireless Sensor Network(WSN) is used.\\ 
In each of cultivation, sensor nodes are deployed to monitor environmental and agricultural parameters. In each region the sensor nodes collected, stored, and transmitted periodically the data to the management node and then the data is sent to the control center and finally the server via the internet. Based at the hardware side are: sensor node, management sensor node and a server.\\
From the data obtained, farmers can observe and decide appropriate actions to control the health of their farm for production quality assurance. The system proposed by the paper is extensible, it improves on precision agriculture and it provides realtime field information.
\subsubsection{Design and Development of Precision Agriculture System Using Wireless Sensor Netwoork}
S. R. Nandurkar, V. R. Thool from the Department of Information Technology Enginnering, SGGIE \& T, Nanded, Nanded(MS) India-431606 worked on a similar project. In their approach, a Wireless Sensor Network(WSN) was used.\\
Their work work was Wireless Sensor Network based low cast soil temperature and moisture monitoring system that can track the soil temperature and moisture of a field in realtime and thereby allow water to be dripped on to the field if the temperature goes above and or the soil moisture falls below a prescribed limit depending on the nature of the crop grown in the soil.The sensors take the inputs like moisture, temperature and provide these inputs to the micro-controller. The micro-controller converts these inputs into the desired form with the program that it is running agive outputs in the mode of regulation of water flow according to the present input conditions. The complete system is implemented for "Smart Irrigation Application" using RF 433MHz modules. The system is designed using a micro-controller and RF 433MHz module.\\ The system provides multiple controls for it users, data collected can be directed towards to an automated irrigation system to trigger irrigation automatically or the farmers can take data and irrigate the farm or field manually.  
\subsubsection{Design and Implementation of a connected Farm for Smart Farming System}
In a paper written by Minwoo Ryu, Taeseok Yun, Ting Miu, $\Pi$-Yeup Ahn, Sung-Chan Choi, Jaeho Kim from the Embedded Software Convergence Research Center, Korean Electronics Technology Institute, Seognam, S. Korea 13509 a connected farm monitoring system was discussed.\\
The goal of the research was provide a suitable environment for growing crops. In this implementation, All sensors and actuators for monitoring and growing crops are connected with a gateway installed a device software platform for IOT systems called \emph{Cube}. The gateway is in turn communicated with an IOT service server called \emph{Mobius}. Accordingly the Mobius not only monitors the environmental condition of the connected farm but also talks with expert farming knowledge systems and controls actuators in order to make the farm suitable to grow crops.\\
The system is extendable, it also easily integrates with new devices and facilitate horizontal smart farm platforms, which enables all smart farms to be connected and take advantage of expert farming knowledge sytems.      
\subsubsection{Design and Implementation of an Agricultural Monitoring System for Smart Farming}
In an article by Jan Bauer and Nils Aschenbruck University of Osnabruck institute of Computer Science from the 2018 IOT vertical and Topic Summit on Agriculture, an agricultural monitoring system for smart farming based on Wireless Sensor Network(WSN) was discussed.\\
This paper was based on a previous paper by the same group. In the previous paper a Photosynthetically Active Radiation(PAR) Sensor was used. The focus of these deployments is on a specific crop parameter, namely the Leaf Area Index(LAI). The LAI is a widely used key parameter that provides information about the photosynthetically performance and vital conditions of plants. The parameter is related to vegetative biomass and simply defined as dimensionless quantity of leaf area of per ground surface area. SInce it also serves as an indicator for yield modeling. The overarching goal of our system is long term continuous crop monitoring that enables LAI profiles with a fine-grained  spatio-temporal resolution. Their previous sensor is used. It senses ambient light in the Photosynthetically Active Radiation(PAR) range. From two simultaneous  PAR measurements; one from below and the other from above the canopy, the transmittance of the irradiation through the canopy can be derived that allows the estimation of the LAI. The key approaches are hardware redundancy, software simplicity, and remote control of the entire system. The architecture developed primarily comprises a WSN-based monitoring system that is tailored for in-situ LAI assessment. The system is fault tolerant due to their hardware redundancy and it is easy to manage since it is remotely controlled.
\subsubsection{Desgn and Implentation of Smart Farm Data Logging and Monitoring System}
Jaina Nica C. Bonquito, Aira Ynnah Luzzyne O. Cabato, Rionel Belen Caldo from the Computer Engineering Department Lyccum of the Phillipines University Lagyna in an article discussed a farm monitoring system for data logging.\\
In their study they designed and implemented a system that log data parameters monitored by different sensors. These four parameters include: temperature(atmospheric), humidity, soil moisture, and soil pH. The sensors used are DHT11 for humidity and temperature, soil moisture sensor and a soil pH sensor. The logged data is integrated by the proponents into a single system which covers effective monitoring of plant growth. The sensed data is sent to the main computer which include VB. Net that is used as a programming language for monitoring aand logging the data values of temperature, humidity,  soil moisture and soil acidity. Microsoft Excel software is used as a database; that is where data is been logged and graphical representation is made.\\
The system implemented is uses a distributed but connected sensors to cover the farm lands. The implementations is good for farms that have the various types of crops grouped in specific places. Different sensors are located are different locations to gather data. The DHT11 is placed at the edges of the farm to monitor temperature and humidity. Each crop-group has it own soil mositure and soil pH sensor which are placed diagonally accross the farm to detect water level and acidity of the soil.\\
This systems allow for change by experimenting strategic locations for the sensors. This is accomplished by the sensors scrutinizing their placements.   
\subsubsection{Design of GreenHouse Environment Monitoring System based on Wireless Sensor Networks}
In an article by Lijun Liu and Yang Zhang from the College of Information Science and Enginnering Shenyang University of Technology Shenyang, China a farm monitoring system for a greenhouse envirnment was discussed.\\
Their is based on Wireless Sensor Network(WSN). This implementation is meant for greehouse farming. The system integrates detection, wireless communication, alarm, display, control and other functions into one, using temperature and humidity sensor(SHT11) and light intensity Sensor(BH1750) for data maonitoring, using CC2530 as microproccessor, man-machine interface is realized by using LabView software.\\ 
The greenhouse envirnment is mainly composed of the monitoring center, the coordinator, the control execution structure and the terminal node. Each function terminal node  transmits environmental information by Zigbee wireless transmission technology to the coordinator via a serial port in the form of cable transmission to the monitoring center. The LabView software is used to display interface of the host computer and display the environmental information of the green house. \\
The system is mobile and flexible, strong expansibility, low cost, low power comsuption and flexible operation.
\subsubsection{Wireless Sensor Network for GreenHouse}


\subsubsection{GreenHouse Monitoring and Control System based on Wireless Sensor Network}

\subsubsection{GreenHouse Monitoring with Wireless Sensor Network}

\subsubsection{A long-term field monitoring system with field sservers at a Grape farm}
In an article by Tokihiro Fukatsu, Yasunori Sairo, Takanobu Suzuki, Kin-chi, Kobayashi and Masayuk Hirufuji and  implementation and an experiment was discussed.\\
The proposed system is constructed using Web-based sensor nodes, an agent program and Web analysis applications, and these modules are connected with each other via the Internet.\\ 
Web-based sensor nodes, one of which was developed as a Field Server in their previous work which has a wireless LAN, an Internet camera, and a monitoring unit with a Web server. A wireless LAN provides high-speed transmission and long-distance communication at low cost, and is therefore effective in monitoring image data.\\
A sensor node equipped with a monitoring unit and an Internet camera, which can be accessed using a Web browser such as Internet Explorer, differs from traditional sensor nodes in its data handling architecture.\\
 The architecture of the Web-based sensor node is of the “pulltype” that is, it functions in the manner of a Web page, showing information only when it is accessed. It functions interactively and performs remote management. Therefore, simply by mounting small firmware on the inside, it can be made to carry out various types of data analysis and complicated operations in the
total system with an agent program and Web applications via the Internet.\\
In this system, each function is separate and the various functions are connected via the Internet, so users at a field site need not manage the system without deploying sensor nodes and connecting them to the Internet. The sensor nodes are controlled by the agent program at a remote site and the collected data is displayed on a Web-based database, whose storage is easily extended and which is accessible to users via the Internet. The agent program, which serves as a cornerstone of this system, accesses and controls all types of Web-based modules seamlessly on the Internet.\\ 
This program operates autonomously based on parameter files (Profiles) in a XML format and it performs complicated operations on the sensor nodes using production rules.\\
It can analyze the monitoring data by using Web applications which execute their process autonomously with input data. By preparing useful Web applications such as image analysis and signal processing, this system provides versatile and easily expansible functions without changing or rebooting the agent program and it makes it possible to distribute calculation tasks.
\subsection{Summary}
\newpage

\section{Methodology}

\end{document}