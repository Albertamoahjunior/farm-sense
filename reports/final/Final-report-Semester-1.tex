\documentclass[12pt, a4paper]{article}
\usepackage{graphicx}
\usepackage{geometry}
\usepackage[utf8]{inputenc}
\graphicspath{{./images}}

\geometry{top=3cm}

\title{KWAME NKRUMAH UNIVERSITY OF SCIENCE AND TECHNOLOGY\\
COLLEGE OF ENGINEERING\\
DEPARTMENT OF COMPUTER ENGINEERING\\ 
\includegraphics[scale=0.4]{{/knust.jpg}} 
\\ALLOCATION OF SENSORS TO MONITOR VITAL SOIL AND ATMOSPHERIC FACTORS THAT AFFECT PLANT GROWTH\\ Case Study: KNUST Peasant Farms}
\author{Amoah Junior Albert - 3022620\\ Bonney Samuel Nii Awuley - 3028320 \\ Lawson Isaac Nii Lante - 3033820 \\ \\ PROJECT SUPERVISOR\\ Dipl. -Ing Benjamin Nii Kommey}

\begin{document}
\maketitle
\thispagestyle{empty}
\newpage

\pagenumbering{roman}
\begin{center}
\textbf{DECLARATION}
\end{center}
This is to certify that the project title \textbf{``Allocation Of Sensors To Monitor Vital Soil And Atmospheric Factors That Affect Plant Growth''} submitted to Kwame Nkrumah University of Science and Technology, represents our original work. The project was conducted under the supervision of Ing. Benjamin Kommey, within the Department of Computer Engineering in the College of Engineering as a partial requirement for the degree of BSc. Computer Engineering.\\ \\ \\  \\ \\ \\
Signature....................................... Date ........................................\\ \\
Amoah Albert Junior(Candidate)\\ \\ \\ \\
Signature....................................... Date ........................................\\ \\
Bonney Nii Awuley(Candidate)\\ \\  \\ \\
Signature....................................... Date ........................................\\ \\
Lawson Isaac Nii Lante(Candidate)\\ \\ \\ \\
Signature....................................... Date ........................................\\ \\
Dipl. -Ing Benjamin Nii Kommey(Supervisor)
\newpage

\begin{center}
\textbf{DEDICATION}
\end{center}
This work is dedicated to our families, friends, and mentors who have provided us with unwavering support and guidance throughout this project. Your encouragement and belief in our abilities have been instrumental in helping us to push through the challenges and achieve our goals. We also dedicate this project to future students who will embark on their own academic journeys, hoping that our work will serve as a source of inspiration and knowledge.
\newpage

\begin{center}
\textbf{ACKNOWLEGEMENT}
\end{center}
Our greatest appreciation goes to God almighty for guiding us successfully through this project. Our appreciation goes to Ing Benjamin Kommey, our supervisor, for providing us with proficient direction and ongoing motivation all through the project. His valuable insights and constructive feedback have played a pivotal role in shaping the trajectory and extent of our efforts.\\
Finally, we would want to express our deepest appreciation to the staff of the Department of Computer Engineering for their support and resources, which have enabled us to complete this project successfully.\\ Thank you all for your contributions to this project and for helping us achieve our academic goals.
\newpage

\begin{center}
\textbf{ABSTRACT}
\end{center}

\newpage


\tableofcontents
\newpage
\pagenumbering{arabic}
\section{Introduction}
\subsection{Background of Study}
Farming is a very important aspect of humanity. Humans cannot live without food and food cannot be without farms, hence the neccessity of farms. Farmers do a great service to man. As the SDG goal number goes: \textbf{Zero Hunger}; End hunger, achieve food security and improved nutrition and promote sustainable agriculture. To achieve this we need to take care of agriculture and improve on its productivity. Agricultural productivity is key to eradicating hunger and achieving food security. As stated in the SDG goal, "promote sustainable agriculture" sustainable agriculture is a much needed achievement. Agricultural productivity can be improved by lay down and implementing measures that maximise crop yield. There are various many factors that affect plant growth and therefore their yield. Plants need certain essentials to grow well and these essentials if provided accurately can increase their growth and therefore increase productivity.
\\
Several factors affect crop health and contribute in the growth of plants soil factors, atmospheric factors and pests\\Soil factors include soil temperature, the soil pH, supply of soil nutrients, soil moisture etc. Atmospheric factors also include atmospheric temperature, humidity, precipitation or rain etc. Soil temperature directly affects plant growth. Most soil organisms function best at an optimum soil temperature. Soil temperature impacts the rate of nitrifi- cation. It also influences soil moisture content, aeration and availability of plant nutrients. Soil pH will influence both the availability of soil nutrients to plants and how the nutrients react with each other. For example: At a low pH, many elements become less available to plants, while others such as iron, aluminum and manganese become toxic to plants. Soil moisture is a measure of soil health, the water content present in a certain area of the ground. All plants need to be in a specific soil moisture range — the majority of plants thrive in soil with a moisture level that ranges between 20 and 60 percent. This is important because if you're trying to grow certain plants, you not only have to make sure the soil is fertile enough to support growth, you have to be able to keep soil moisture in a certain range. It's an issue for industrial growers, but also private gardeners and anyone trying to grow their own vegetables. This is because the water content in soil is a solvent, meaning that it breaks down the nutrients and minerals that plants need from the dirt, allowing them to absorb these helpful particles into their systems. As atmospheric temperature increases (up to a point), photosynthesis, transpiration and respiration increase. When combined with day length, temperature also affects the change from vegetative (leafy) to reproductive (flowering) growth. Too much rain during germination may saturate soils, resulting in poor germination and reduced stands. However, too little rain during germination may reduce germination and leave plants ill-prepared for future growth and development challenges. When relative humidity levels are too high or there is a lack of air circulation, a plant cannot make water evaporate (part of the transpiration process) or draw nutrients from the soil. When this occurs for a prolonged period, a plant eventually rots.\\ \\
From the above study we can infer that realtime and accurate information on the condition of a farm in needed.
\subsection{Problem Statement}
Agricultural farms are important to humans. They produce the food we eat. The food we eat are the yields of the crops that are cultivated.\\
Farmers frequently face challenges in providing the necessary farm essentials to support ideal plant growth, primarily due to inadequate and inaccurate farm condition data. The problem is that farmers have been providing farm essentials to crops based on guess work. They are not able to accurately tell the needs of the farms.
\\
Current farm management practices are often reactive and data-deficient, leading to inefficiencies and unsustainable resource use across various aspects of agricultural production Farmers struggle to optimize irrigation, fertilization, and pest control due to limited real-time data on soil health, weather conditions, and crop development, resulting in potential yield losses, water waste, and environmental damage from excess nutrients.
\\
Therefore, there is a critical need for a farm monitoring system that:
\begin{itemize}
 \item Provides real-time, comprehensive data on environmental conditions, crop health.
 \item Offers user-friendly interfaces and actionable insights for informed decision-making.
 \item Is cost-effective, scalable, and accessible to farmers of all sizes.
 \item Contributes to sustainable agriculture practices by optimizing resources
\end{itemize}
By addressing these challenges a well-designed system that can monitor vital soil and atmospheric factors that affect farm plant growth can empower farmers to achieve: increased productivity and profitability, enhanced decision-making and long-term sustainability.



\newpage
\subsection{Objectives of the Project}
\subsubsection{General Objectives}
The project's objective is to develop a streamlined approach for deploying sensors to monitor critical soil and atmospheric conditions essential for plant growth. This involves identifying the optimal quantity, types, and placements of sensors to collect data on variables such as moisture levels, nutrient content, temperature, humidity, and atmospheric gases. Ultimately, the aim is to equip farmers with timely information to refine their farming techniques, resulting in higher crop yields and quality, decreased resource consumption, and a reduced environmental impact.
\subsubsection{Specific Objectives}
This project aims to achieve several key objectives:\\
\begin{enumerate}
\item \textbf{Increased efficiency and productivity:}
\begin{enumerate}
 \item Optimize resource use (water, fertilizer, energy) based on real-time data.
 \item Automate tasks like irrigation and climate control to save time and labor.
 \item Improve yield and crop quality through early detection of problems.
\end{enumerate}
 
\item \textbf{Reduced environmental impact:} Minimize water usage and fertilizer runoff through precise application.\\
\item \textbf{Improved decision-making:} Facilitate informed decision-making for improved farm management and profitability.
\end{enumerate}

\newpage
\subsection{Scope of the project}
The focus of this project is to design a system that can be used to monitor any size of open farms.
\begin{itemize}
\item[--] The system will measure specific soil and atmospheric factors. Atmospheric factors include: atmospheric temperature, humidity, and rain/precipitation. Soil factors also include: soil pH, soil moisture, soil temperature and supply of specific nutrients.
\item[--] The whole system is a Wireless Sensor Network(WSN) based implementation been used to  monitor the farm.
\item[--] The system will also take the information gathered via the sensors and then sent to a cloud service for processing. Based on the information derived, farmers will be advised on what to do.
\item[--] The system will have a frontend interface where information derived can be displayed and farmers recieve advise.
\item[--] The system will also offer sms service where sms can be sent to farm owners of the conditions of their farm. 
\end{itemize}

\newpage
\subsection{Significance of Study}
Allocation of sensors to monitor vital soil and atmospheric factors is very essential. The can help improve on our agricultural prowess and productivity. 
\\ \\
\textbf{Optimization of Resources:} The system can help farmers optimize the use of resources such as water, fertilizers, and pesticides, reducing waste and increasing efficiency.
\\ \\
\textbf{Early Detection of Issues:} This system can detect issues such as pest infestations, disease outbreaks, or nutrient deficiencies early, allowing farmers to take timely action to mitigate losses.
\\ \\
\textbf{Data-Driven Decision Making:} By collecting and analyzing data on crop health, weather patterns, soil moisture, and other factors, farmers can make informed decisions to improve yields and profitability.
\\ \\
\textbf{Remote Monitoring:} With this system, farmers can remotely monitor their fields allowing them to keep an eye on their operations even when they are not physically present.
\\ \\
\textbf{Environmental Sustainability:} By monitoring factors like soil health and water usage, farmers can adopt more sustainable practices that minimize environmental impact and conserve natural resources.
\\ \\
\textbf{Increase in Productivity:} By providing real-time insights into crop conditions, monitoring systems can help farmers increase productivity and ultimately profitability.


Overall, farm monitoring systems enable farmers to make more informed decisions, increase efficiency, and ultimately improve the sustainability and profitability of their operations.

\newpage
\subsection{Organisation of Study}
\begin{enumerate}
\item \textbf{Introduction}
	\begin{itemize}
	\item[--] Aim and objectives of the project.
	\item[--] Background information on how certain soil and atmospheric factors affect plant    growth.
	\end{itemize}
\item \textbf{Farm monitoring systems}
\begin{itemize}
    \item[--] Types of farm monitoring systems implemented.
	\item[--] Benefits and limitations of existing technology.
\end{itemize}
\item \textbf{Design of System}
\begin{itemize}
	\item[--] Design of system architecture.
    \item[--] Physical design consideration for the system.
    \item[--] Components of the smart monitoring system.
\end{itemize}
\item \textbf{Programming of the System}
\begin{itemize}
	\item[--] Appropriate programming laguages that can be used for programming the system and controling the various components.
	\item[--] Algorithims that would be used to gather and analyze information.
\end{itemize}
\item \textbf{Prototype development}
\begin{itemize}
\item[--] Building a prototype of the monitoring system
\item[--] Testing the prototype for efficiency and functionality
\item[--] Iterative improvement of the prototype system
\end{itemize}
\item \textbf{Conclusion}
\begin{itemize}
\item[--] Summary of the study.
\item[--] Future scope of the project.
\end{itemize}
\end{enumerate}

\section{Literature Review}
\subsection{Introduction}
Over the past few years, the need for precision and accurate agriculture has been on the rise. This is all because it has been found out that resources are wasted or not used adequately during farming. For instance water; farms could easily be under-irrigated or over-irrigated and go on unnoticed, Fertilizers could be overused or underused. This is all because there is no way of accurately telling the condition of the soil the plant is embedded in or the environmental factors that affect it. Precision agriculture eliminates guess work, encourages data-driven decisions and helps mitigate the wastage of resources. Due to these advantages in the field of agriculture, there has been the proliferation of serveral technologies to solve this problem efficiently. In this section of the report, articles on studies and implementations of such technologies are discussed.

\subsection{Related Works}
\subsubsection{Design and Deploy a Wireless Sensor Network for Precision Agriculture}
In an article by Tuan Din Le and Dat Ho Tan \cite{7302210} from Department of Computer Science Long An University of Economics and Industry in 2015 during the second National Foundation for Science and Technology Development Conference on Information and Computer Science a similar project was discussed. In their approach, a Wireless Sensor Network(WSN) is used.\\ 
In each of cultivation, sensor nodes are deployed to monitor environmental and agricultural parameters. In each region the sensor nodes collected, stored, and transmitted periodically the data to the management node and then the data is sent to the control center and finally the server via the internet. Based at the hardware side are: sensor node, management sensor node and a server.\\
From the data obtained, farmers can observe and decide appropriate actions to control the health of their farm for production quality assurance. The system proposed by the paper is extensible, it improves on precision agriculture and it provides realtime field information.

\subsubsection{Design and Development of Precision Agriculture System Using Wireless Sensor Netwoork}
S. R. Nandurkar, V. R. Thool \cite{6808017} from the Department of Information Technology Enginnering, SGGIE \& T, Nanded, Nanded(MS) India-431606 worked on a similar project. In their approach, a Wireless Sensor Network(WSN) was used.\\
Their work work was Wireless Sensor Network based low cast soil temperature and moisture monitoring system that can track the soil temperature and moisture of a field in realtime and thereby allow water to be dripped on to the field if the temperature goes above and or the soil moisture falls below a prescribed limit depending on the nature of the crop grown in the soil.The sensors take the inputs like moisture, temperature and provide these inputs to the micro-controller. The micro-controller converts these inputs into the desired form with the program that it is running agive outputs in the mode of regulation of water flow according to the present input conditions. The complete system is implemented for ``Smart Irrigation Application" using RF 433MHz modules. The system is designed using a micro-controller and RF 433MHz module.\\ The system provides multiple controls for it users, data collected can be directed towards to an automated irrigation system to trigger irrigation automatically or the farmers can take data and irrigate the farm or field manually.  

\subsubsection{Design and Implementation of a connected Farm for Smart Farming System}
In a paper written by Minwoo Ryu, Taeseok Yun, Ting Miu, $\Pi$-Yeup Ahn, Sung-Chan Choi, Jaeho Kim \cite{7370624} from the Embedded Software Convergence Research Center, Korean Electronics Technology Institute, Seognam, S. Korea 13509 a connected farm monitoring system was discussed.\\
The goal of the research was provide a suitable environment for growing crops. In this implementation, All sensors and actuators for monitoring and growing crops are connected with a gateway installed a device software platform for IOT systems called \emph{Cube}. The gateway is in turn communicated with an IOT service server called \emph{Mobius}. Accordingly the Mobius not only monitors the environmental condition of the connected farm but also talks with expert farming knowledge systems and controls actuators in order to make the farm suitable to grow crops.\\
The system is extendable, it also easily integrates with new devices and facilitate horizontal smart farm platforms, which enables all smart farms to be connected and take advantage of expert farming knowledge sytems.    
  
\subsubsection{Design and Implementation of an Agricultural Monitoring System for Smart Farming}
In an article by Jan Bauer and Nils Aschenbruck \cite{8373022} University of Osnabruck institute of Computer Science from the 2018 IOT vertical and Topic Summit on Agriculture, an agricultural monitoring system for smart farming based on Wireless Sensor Network(WSN) was discussed.\\
This paper was based on a previous paper by the same group. In the previous paper a Photosynthetically Active Radiation(PAR) Sensor was used. The focus of these deployments is on a specific crop parameter, namely the Leaf Area Index(LAI). The LAI is a widely used key parameter that provides information about the photosynthetically performance and vital conditions of plants. The parameter is related to vegetative biomass and simply defined as dimensionless quantity of leaf area of per ground surface area. SInce it also serves as an indicator for yield modeling. The overarching goal of our system is long term continuous crop monitoring that enables LAI profiles with a fine-grained  spatio-temporal resolution. Their previous sensor is used. It senses ambient light in the Photosynthetically Active Radiation(PAR) range. From two simultaneous  PAR measurements; one from below and the other from above the canopy, the transmittance of the irradiation through the canopy can be derived that allows the estimation of the LAI. The key approaches are hardware redundancy, software simplicity, and remote control of the entire system. The architecture developed primarily comprises a WSN-based monitoring system that is tailored for in-situ LAI assessment. The system is fault tolerant due to their hardware redundancy and it is easy to manage since it is remotely controlled.

\subsubsection{Desgn and Implentation of Smart Farm Data Logging and Monitoring System}
Jaina Nica C. Bonquito, Aira Ynnah Luzzyne O. Cabato, Rionel Belen Caldo \cite{bongulto2016design} from the Computer Engineering Department Lyccum of the Phillipines University Lagyna in an article discussed a farm monitoring system for data logging.\\
In their study they designed and implemented a system that log data parameters monitored by different sensors. These four parameters include: temperature(atmospheric), humidity, soil moisture, and soil pH. The sensors used are DHT11 for humidity and temperature, soil moisture sensor and a soil pH sensor. The logged data is integrated by the proponents into a single system which covers effective monitoring of plant growth. The sensed data is sent to the main computer which include VB. Net that is used as a programming language for monitoring aand logging the data values of temperature, humidity,  soil moisture and soil acidity. Microsoft Excel software is used as a database; that is where data is been logged and graphical representation is made.\\
The system implemented is uses a distributed but connected sensors to cover the farm lands. The implementations is good for farms that have the various types of crops grouped in specific places. Different sensors are located are different locations to gather data. The DHT11 is placed at the edges of the farm to monitor temperature and humidity. Each crop-group has it own soil mositure and soil pH sensor which are placed diagonally accross the farm to detect water level and acidity of the soil.\\
This systems allow for change by experimenting strategic locations for the sensors. This is accomplished by the sensors scrutinizing their placements. 
  
\subsubsection{Design of GreenHouse Environment Monitoring System based on Wireless Sensor Networks}
In an article by Lijun Liu and Yang Zhang \cite{liu2017design} from the College of Information Science and Enginnering Shenyang University of Technology Shenyang, China a farm monitoring system for a greenhouse envirnment was discussed.\\
Their is based on Wireless Sensor Network(WSN). This implementation is meant for greehouse farming. The system integrates detection, wireless communication, alarm, display, control and other functions into one, using temperature and humidity sensor(SHT11) and light intensity Sensor(BH1750) for data maonitoring, using CC2530 as microproccessor, man-machine interface is realized by using LabView software.\\ 
The greenhouse envirnment is mainly composed of the monitoring center, the coordinator, the control execution structure and the terminal node. Each function terminal node  transmits environmental information by Zigbee wireless transmission technology to the coordinator via a serial port in the form of cable transmission to the monitoring center. The LabView software is used to display interface of the host computer and display the environmental information of the green house. \\
The system is mobile and flexible, strong expansibility, low cost, low power comsuption and flexible operation.

\subsubsection{Wireless Sensor Network for GreenHouse}
An article by S.U Zagade, R.S Kawitkar \cite{zagade2012wireless} from the Department of E \& TC, Singhad College of Enginnering, Pune, India describes how a Wireless Sensor Network was used to monitor a green house farm.\\
In their approach all monitored parameters are transmitted through a  wireless link to cellular device for analysis. A cell phone is used instead of computer termianls noting that farmers at the ones going to use the system and power management also. There are a total of three sensor nodes; each sensing: temperature, humidity and light intensity in addition to general purpose computing and networking devices.\\
The computation module on each sensor node is a programmable unit that performs computation and bi-directional communication with other sensor node. It interfaces with the digital sensors on the sensor module and dispatches the data according to the application needs. Since the wireless communication range of Radio Frequency(RF) module is more than 1km, the sensor node can be widely separated. Sensor node 1 and sensor node 2 transmit their data through the wireless communication link to the sensor node 3 which acts as a  cordinator node aggrgates the data in time multiplexed manner, which helps in avoiding collision of data transmission. Coordinator node also acts as a gateway node between two different wireless technologies. It transmits its collected data along with its own data to the cell phone using Short Message Services(SMS). The cell phone is used instead of computer terminal to increase the distance to create simplicity in a network as well as to minimize the power consumption. The network formed by node 1 and node 2 is known as patch network.\\
The advantage of making node 3 coordinator as well as gateway node is to increase area covered by the system. The caveat is that this system has a single point of failure, if the the coordinator node goes down the whole system becomes ineffective.  

\subsubsection{GreenHouse Monitoring and Control System based on Wireless Sensor Network}
In an article by Marwa Mekki, Osman Abdallah, Magdi B. M. Amin, Moez Eltayeb, Tafaoul Abdalfatah, Amin Babiker \cite{7381396} from Sudan Atomic Energy Commission, Faculty of Engineering and Technology, University Of Gezira, Wad Medani-Sudan Faculty of Engineering, Alneelain University Khartoum-Sudan during the international conference on Computing, Control, Networking, Electronics and Embedded Systems, a similar project was discussed.\\
The system consists of a temperature sensor, humidity sensor, moisture sensor, light control and C02 sensor. The various sensors monitor various parameters and when they go beyond or above, they trigger actuators to curb the situation. All the parameters are user defined, plan dependent and climate requirements.\\
The sensors values are sent to the gateway, the node checks it and the global data required by checking a new sms arrival. If recieved, the node sends an SMS data frame to a master mobile phone with the sensor values. The frame contains the sensor abbreviation and the value. The node also checks if a control SMS recieved. In this case, the system responds to the frame contents by decoding the SMS and turning ON/OFF a device according to the frame contents. The control data frame contain an abbreviation for a device and status needed. The data could be sent by master mobile phone or from the GUI on demand. If the sensors run off limits, the system sends a caution SMS to the master mobile phone to take a decision. If the mobile phone responds, the system will obey otherwise the sytem will automatically trigger a device.\\
In the gateway side, the system recieves the sensors' data and send them to the LabView software. The software displays the sensors' values also has control switches for the different devices. The control data frame could be sent to the sensor node using Devices Control Switch(DCS).  
 
\subsubsection{GreenHouse Monitoring with Wireless Sensor Network}
A team of three from the University of Vassa: Teemu Ahonen, Reino Virranskoski and Mohammed Elmusrati \cite{torabi2023greenhouse} worked on a greenhouse farm monitoring system.\\
In their work they integrated three commercial sensors to the sensinode's sensor platform. By using these sensors they were able to measure four parameters which are crucial in greenhouse climate adjustment: temperature, relative humidity, light irradiance and air carbon dioxide content.\\
The platform 6LoWPAN protocol, which allows the sending of compressed IPV6 packets over IEEE 802.15.4 networks. The sensor nodes communicate directly to the gateway which acts as a coordinator and recieves measured data.  A computer is then connected to the coordinator by a usb-cable. 

\subsubsection{A long-term field monitoring system with field sservers at a Grape farm}
In an article by Tokihiro Fukatsu, Yasunori Sairo, Takanobu Suzuki, Kin-chi, Kobayashi and Masayuk Hirufuji \cite{saito2008long} implementation and an experiment was discussed.\\
The proposed system is constructed using Web-based sensor nodes, an agent program and Web analysis applications, and these modules are connected with each other via the Internet.\\ 
Web-based sensor nodes, one of which was developed as a Field Server in their previous work which has a wireless LAN, an Internet camera, and a monitoring unit with a Web server. A wireless LAN provides high-speed transmission and long-distance communication at low cost, and is therefore effective in monitoring image data.\\
The sensor nodes are controlled by the agent program at a remote site and the collected data is displayed on a Web-based database, whose storage is easily extended and which is accessible to users via the Internet.\\ 
This program operates autonomously based on parameter files (Profiles) in a XML format and it performs complicated operations on the sensor nodes using production rules.\\
It can analyze the monitoring data by using Web applications which execute their process autonomously with input data. By preparing useful Web applications such as image analysis and signal processing, this system provides versatile and easily expansible functions without changing or rebooting the agent program and it makes it possible to distribute calculation tasks.

\subsubsection{An Efficient Wireless Sensor Network for Precision Agriculture}
In an article by Manijeh Keshtgary, Amene Deljoo \cite{keshtgary2012efficient} discussed a farm monitoring and control system based on Wireless Sensor Network\\
In their approach the system consists of devices(ie. sensors) spread in an environment in order to monitor and manage it, based on the specific physical phenomena. By using computer resources and appropriate technology, such activities are done automatically. It is designed such that proper decision can be made for each zone in the farm. Unlike traditional networks, which assume the user to be a human agent, WSN are centered in the physical environment, specially the data themselves. The sensor nodes interact with the environment on which they are inserted, capturing information about it based on interesting physical phenomena and collaborating among themselves, helping to do tasks that need to be done in this method. In order to achieve these goals, the use of specific algorithms and communication protocols are essential, once the nodes are distributed in the environment and need to self configure the network and adapt themselves to it. These sensors can be programmed to record measures like temperature and humidity. All the data which are collected from the sensors, using a wireless multi-hop routing technology, end up in a sink node which transfers them to the end user through wireless network, internet or LAN. The system plays the following roles: sensing agricultural parameters, identification of sensing location and data gathering, transferring data from crop field to control station for decision making and Actuation and Control decision based on sensed data.

\subsubsection{PRELIMINARY DESIGN FOR CROP MONITORING INVOLVING WATER AND FERTILIZER CONSERVATION USING WIRELESS SENSOR NETWORKS}
A research paper by S. Vijayakumar, J. Nelson Rosario \cite{vijayakumar2011preliminary} from TIF AC-CORE in Pervasive Computing Technologies, Velammal Engineering College, Chennai, India aimed to develop a system that help curb water and fertilizer wastage.\\
In their work, they propose a wireless sensor system that will communicate with each other with low power consumption. This is done using Micaz motes. The architecture then to be implemented in the sensor nodes will construct a wireless networking data collection at crop field likely to replace the conventional manually data collection system. A general Micaz mote with MDA300 data acquisition board has standard measurement parameters sensors such as ambient air temperature and humidity and also has external terminals for soil pH, soil moisture, leaf wetness and atmospheric pressure sensors all to be integrated in all nodes. All the deployed nodes will collect the parameters and report to the central co-ordinator/sink. The coordinator will coordinate the data collection. The individual nodes based on the soil moisture sensor content attached to it will excite the water sprinklers in that particular region. Meanwhile the soil pH sensor value will be reported to the central coordinator, and then the soil pH value is reported to the farmer using SMS system via OSM modem indicating him to fertilize the particular region. This proposal seem to help conserve water and fertilizer.

\subsubsection{A Strategic Agricultural Field Monitoring using Internet of Things enabled Wireless Sensor Network}
This paper by Timothy Dhayakar Paul from the Department of ECE at Kumaraguru College of Technology, Coimbatore, India, and Dr. Vimalathithan Rathinasabapathy from the Department of ECE at Karpagam College of Engineering, Coimbatore, India \cite{paul2021strategic} discusses the implementation of a novel Wireless Sensor Network (WSN) based Agricultural Management System with the integration of Internet of Things (IoT) technology.\\
Referred to as the Intellectual Agri-Data Processing Scheme (IADPS), this approach aims to provide real-time monitoring and management of agricultural fields. The system collects data such as temperature, humidity, soil moisture, and motor pump condition using sensors placed in the agricultural field. This data is then transmitted to a server for processing and storage. Alerts are generated for farmers when the data exceeds predefined threshold levels, facilitating timely interventions to ensure crop health and productivity.\\
The process involves the deployment of IoT-assisted WSN using a Smart Device equipped with sensors for data collection. The data is transmitted to a server via the WSN base station for processing. The system employs algorithms to analyze the data and trigger alerts based on predefined thresholds. The proposed approach aims to enhance agricultural field monitoring and management by leveraging IoT and WSN technologies.\\
The integration of IoT and WSN technologies provides a comprehensive solution for real-time agricultural field monitoring. The use of sensors enables accurate data collection, allowing for timely interventions to optimize crop health and productivity. The system's ability to send alerts to farmers promptly helps in preventing potential crop damage or losses. The paper provides a detailed explanation of the proposed methodology, including algorithms and system architecture.\\
In spite of these great benefits, paper lacks empirical data or case studies to demonstrate the effectiveness of the proposed system in real-world agricultural settings. There is limited discussion on potential challenges or limitations of implementing the proposed approach. The scalability and cost-effectiveness of deploying the system on a large scale are not addressed. 

\subsubsection{Smart farm and monitoring system for measuring the Environmental condition using wireless sensor network-IOT Technology in farming}
In a paper published in the 2020 5th International Conference on Innovative Technologies in Intelligent Systems and Industrial Applications by Tharindu Madushan Bandara and Mansoor RAZA\cite{bandara2020smart} a solution was proposed.\\
In their approach they used IoT sensors, including soil moisture sensors, temperature sensors, and
water volume sensors, to collect data from the farming environment.  Wireless sensor networks (WSNs) were employed to transmit data from sensor nodes to the central server. Analysis of the collected data was performed on the central server to monitor and control environmental conditions in real-time.\\
The advantages of the system are that, it utilizes modern IoT technology to improve farming efficiency and resource management. The Wireless sensor networks enable real-time data collection from sensor nodes placed in the farming environment. The system also provide a reliable and flexible solution for farmers with a simple architecture. Incorporates a real-time monitoring system for better control of environmental conditions.\\
However, The paper lacks detailed information on the specific implementation and deployment of the
proposed system. It also has limited discussion on potential challenges and limitations of the system.
Again it may require further validation and testing in real-world farming scenarios.

\subsubsection{Light Control Smart Farm Monitoring System with Reflector Control}
A research conducted by Jaekuk Choi, Dongsun Lim, Sangwon Choi, Jeonghyeon Kim, and Jonghoek Kim \cite{choi2020light} from the Department of Electrical Engineering at Hongik University, Sejong, Korea discussed discussed a farming monitoring system with a reflector control.\\
In their approach, they implemented a smart farm monitoring and control system using Arduino and DC motors. They controlled the amount of light inflow by adjusting the angle of a reflector. Environmental factors such as temperature, humidity, carbon dioxide (CO2), and light were monitored for optimal farm conditions. The system included a server and mobile application for real-time data upload and monitoring.\\
The implementation has some interesting pros, the approach of using reflector control instead of artificial light for cost-effective and efficient light management in smart farms is novel.  The system comprehensively monitors various environmental factors crucial for plant growth. Remote modification of temperature and humidity reference values via a mobile application enhances control flexibility.
Accumulated data on the server enables analysis for maintaining optimal farm conditions.
\\
However, there are some caveats: the paper lacks detailed experimental results: including extensive data analysis or statistical validation of the system's performance, the scalability of the system to larger farms or different environments is not discussed.

\subsubsection{IoT Enabled Plant Soil Moisture Monitoring Using Wireless Sensor Networks}
A.M. Ezhilazhahi and P.T.V. Bhuvaneswari \cite{ezhilazhahi2017iot} from the Department of Electronics Engineering, Madras Institute of Technology, conducted research on developing a remote monitoring system for soil moisture in plants using Wireless Sensor Networks (WSN) integrated with Internet of Things (IoT) technology.\\
In their approach, he researchers developed a system consisting of sensing and transmitter modules,
receiver module, IoT enablement using Dropbox cloud storage, and an event detection algorithm based on Exponential Weighted Moving Average (EWMA). The sensing module included a soil moisture detection probe and sensor board connected to a PIC microcontroller and Zigbee transceiver. The receiver module utilized a Raspberry Pi and Zigbee receiver.\\
On the bright side, the research addresses the increasing demand for continuous monitoring of plant health, particularly soil moisture, in organic farming. There is te integration of WSN and IoT technologies offers remote monitoring capabilities. The Zigbee technology also  used ensures efficient wireless communication. The adoption of the EWMA event detection algorithm enhances the network lifetime by activating nodes only when threshold conditions are met, conserving energy.\\
On the other side, The research primarily focuses on monitoring soil moisture in an open environment rather than controlled environments like greenhouses, potentially limiting its applicability in certain
agricultural contexts. While the system's components are described, specific technical details such as sensor accuracy, power consumption, and data transmission protocols are not extensively discussed. Limited discussion on scalability or robustness of the proposed system beyond single-sensor deployments.

\subsubsection{Design of Novel Wireless Sensor Network Enabled IoT based Smart Health Monitoring System for Thicket of Trees}
This work was conducted by B. Sridhar, S. Sridhar, and V. Nanchariah. B. Sridhar and S.Sridhar \cite{sridhar2020design} from the Department of Electronics and Communication Engineering at LIET, Vizianagaram, India, while V. Nanchariah is an associate professor in the same department.\\
In their approach, The literature discusses the design and implementation of a novel wireless sensor network-enabled IoT-based smart health monitoring system specifically tailored for the monitoring of
coconut trees. The system aims to address issues such as natural calamities, diseases like red spot
disease, and the lack of proper monitoring systems, which have led to a decline in coconut farming in countries like India.\\
The proposed system utilizes a combination of wireless sensor networks (WSN) and IoT technologies to monitor and control the health status of coconut trees remotely. It incorporates cloud-based servers and mobile devices for real-time monitoring and control. The system integrates soil moisture monitoring using WSN to achieve its objectives.\\
Positively, the system offers remote monitoring and control of tree health, providing convenience to
farmers, it utilizes IoT and WSN technologies, which are known for their efficiency in data
collection and monitoring, the integration of cloud-based servers ensures accessibility and scalability of the system low power consumption and solar power compatibility make the system suitable for remote and rural areas.\\
Contrawise, The literature lacks detailed information on the specific sensor types and their accuracy.
It does not provide empirical data or results from field trials to validate the effectiveness of the proposed system. While the paper discusses technical challenges, it does not delve deeply into solutions or mitigation strategies for these challenges. The scalability and robustness of the system in large-scale deployment scenarios are not thoroughly addressed. 

\subsubsection{Greenhouse Monitoring with Biocompatible Humidity Sensor for Smart Farming}
A study made by Moritz Schlagmann, Joseph Stoenner, Franz Selbmann, Stefan Hess, Thomas Otto \cite{schlagmann2023greenhouse} discussed a similar project.\\
This was a study on greenhouse monitoring using biocompatible humidity sensors for smart farming. The research addresses the need for precise environmental monitoring in greenhouse horticulture to optimize crop growth and mitigate pest-related risks.\\
 The study utilizes a miniaturized leaf wetness sensor integrated with an application-specific integrated circuit (ASIC) based on CMOS technology. The sensor employs interdigitated electrodes covered with biocompatible Parylene C, allowing direct attachment to plant leaves. Calibration of the sensor is conducted to ensure accurate humidity measurements. Also there is the development of  an algorithm to determine leaf wetness duration based on dew point depression (DPD) calculated from sensor data. Different DPD thresholds are evaluated to optimize wetness detection accuracy.
 
 \subsubsection{Design of a smart system for monitoring and management of pastures and meadows: The Relational Database Approach}
This study was conducted by Tsvetelina Mladenova, Irena Valova, and Nikolay Valo \cite{mladenova2022design} from the University of Ruse, Bulgaria.\\
The study proposes a system comprising both hardware and software components for monitoring and managing smart farms. The hardware aspect involves the deployment of measuring stations equipped with various sensors, microcontrollers, batteries, and communication modules. The software part includes modules for data collection, processing, normalization, relational database management, user interface, and machine learning model training.\\
Some pros derived are: The study addresses the growing interest and importance of ICT in agriculture, specifically in the context of smart farming solutions. The proposed system integrates hardware and software components for comprehensive farm monitoring and management. Utilization of a relational database approach ensures efficient data storage, processing, and analysis. Detailed descriptions of hardware components and software functionalities provide clarity on system implementation.\\
However there are also some caveats: The study primarily focuses on the technical aspects of the proposed system, with limited discussion on potential socio-economic impacts or scalability issues. While the software functionalities are described in detail, practical implementation challenges and potential barriers to adoption are not extensively discussed. Further empirical validation or field testing of the proposed system's performance and scalability could enhance the study's credibility.

\subsubsection{Cloud-based Low-Power Long Range IoT Network for Soil Moisture Monitoring in Agriculture}
Subhra Shankha Bhattacherjee, Shreeshan S., Gattu Priyanka, Akshay Ramesh Jadhav, P. Rajalakshmi (Indian Institute of Technology, Hyderabad, India), Jana Kholova (Crop Physiology, ICRISAT, Hyderabad, India)\cite{bhattacherjee2020cloud} conducted a study soil moisture monitoring in agricultural.
In their study, the proposed IoT network utilizes LoRa (Long Range) communication technology operating
at 868 MHz ISM band for transmitting data. The authors have designed soil moisture sensors and LoRa nodes in-house. These sensors measure various field parameters including ambient temperature, humidity, soil moisture, and soil temperature. The nodes are powered by rechargeable Li-Ion batteries and solar panels, enhancing their longevity. The network architecture incorporates Raspberry Pi as a gateway device, connecting to servers hosted by OVH, a cloud computing company.\\
The paper addresses the critical need for efficient soil moisture monitoring in agriculture,
emphasizing the benefits of IoT technology in addressing this need. The use of LoRa technology enables long-range communication, making it suitable for large agricultural fields. The in-house design of sensor nodes and LoRa communication devices demonstrates technical expertise and customization according to specific requirements. The deployment of the network at ICRISAT's LeasyScan platform validates its practical applicability in real-world agricultural settings.\\
While the study focuses on soil moisture monitoring, it does not extensively discuss other relevant field parameters such as nutrient content, which could also impact crop yield. The paper lacks detailed discussion on the data analysis techniques employed, such as statistical analysis to address outliers in the collected data. Limited information is provided regarding the scalability of the proposed network and its compatibility with diverse agricultural landscapes.

\subsubsection{Connectivity of Agricultural Soil Fertility Online Monitoring Devicess with Smartphones}
The research was conducted collaboratively by Denis Eka Cahyani, Langlang Gumilar, Arif Nur Afandi, Ahmad Asri Bin Abd Samat, Achmad Safi'i, and M. Wahyu Prasetyo \cite{cahyani2023connectivity} from the Departments of Mathematics and Electrical Engineering at Universitas Negeri Malang, Indonesia, and Universiti Teknologi Mara, Malaysia.\\
The study presents the development of an online monitoring system for agricultural soil fertility using Internet of Things (IoT) technology. The system integrates sensors for soil pH, temperature, and moisture connected to a microcontroller, which transmits data to smartphones for real-time monitoring.\\
The system employs sensors including DS18B20 for temperature, YL69 for moisture, and pH sensors, integrated with a NodeMCU ESP32 microcontroller. The microcontroller processes data and sends it to smartphones via the Internet, allowing real-time monitoring. The study also describes the operational principles and testing procedures for each sensor.\\
The integration of IoT technology allows for real-time monitoring of soil fertility, aiding farmers in timely decision-making. The study provides detailed descriptions of sensor functionalities, testing methods, and system architecture, ensuring reproducibility. The system utilizes widely accessible components, such as smartphones, making it potentially scalable and adaptable in various agricultural settings.\\
The study primarily focuses on sensor accuracy testing, with limited discussion on the system's scalability, robustness, or potential limitations in practical implementation. While the system's connectivity with smartphones is demonstrated, further discussion on the user interface, data visualization, and usability aspects could enhance its applicability for farmers. The research lacks empirical validation in diverse agricultural environments, which could affect the generalizability of the findings.

\subsubsection{IoT Based Greenhouse Environment Monitoring and Smart Irrigation System for Precision Farming Technology}
In a paper by Arindom Chakraborty, Mohammad Shahadat Hossain, Monirul Islam, Animesh Dhar \cite{chakraborty2022iot} they discussed Greenhouse environment farm monitoring system.
The proposed IoT-based precision farming system, divided into three subsystems: local monitoring, IoT monitoring, and smart irrigation. It details the components used, such as sensors, microcontrollers, and the BLYNK IoT platform.\\
The study implemented the proposed system in a small-scale greenhouse model. Sensors were placed inside the greenhouse to measure temperature, humidity, light intensity, and soil moisture. The system's performance, including the smart irrigation functionality, was tested and validated.\\
The results demonstrated the system's effectiveness in real-time monitoring and control of environmental parameters. The smart irrigation system successfully maintained soil moisture levels, enhancing resource efficiency and crop growth.\\
The study has comprehensive literature review showcasing the evolution of precision farming and IoT applications, Clear description of the proposed IoT-based precision farming system and its implementation, successful demonstration of the system's performance in real-world conditions, emphasis on affordability and user-friendliness, catering to the needs of farmers.\\
However, it has limited discussion on the scalability and long-term sustainability of the proposed system,
absence of comparative analysis with existing solutions in terms of cost-effectiveness and performance, future directions could be further elaborated for a more comprehensive roadmap.

\subsubsection{Study on Precision Agriculture Monitoring Framework Based on WSN}
This is a study by Xuemei Li (Institute of Built Environment and Control, Zhongkai University of Agriculture and Engineering, Guangzhou, China; South China University of Technology), Yuyan Deng, Lixing Ding (Institute of Built Environment and Control, Zhongkai University of Agriculture and Engineering, Guangzhou, China)\cite{li2008study}.
The study focuses on the implementation of a Wireless Sensor Network (WSN) based greenhouse management system to facilitate modern precision agriculture. It describes the deployment of WSN in greenhouse environments to monitor various parameters, control equipment, and offer services to consumers. The system aims to achieve cost-effectiveness and environmental friendliness in greenhouse management.\\
The study implements a WSN-based greenhouse management system to monitor greenhouse environments and control equipment. It utilizes sensor technology, micro-electromechanical systems (MEMS), wireless communication, embedded computing, and distributed information management technologies. The system architecture involves sensor nodes, actuator nodes, sink nodes, gateways, and a management sub-system.\\
 Integration of various technologies: The study effectively integrates sensor technology, wireless communication, and computing to develop a comprehensive greenhouse management system.
The implementation of the system in greenhouse environments demonstrates practical applicability. The study emphasizes the advantages of using WSN for greenhouse management, highlighting its cost-effectiveness and environmental benefits.

\subsubsection{Web-Based Smart Farm Data Monitoring System:Prototype}
The study is conducted by Onine M. Mico, Paul Bryan M. Santos, and Rionel B. Caldo \cite{mico2016web} from the Computer Engineering Department of Lyceum of the Philippines University - Laguna.\\
The paper presents a prototype for a web-based smart farming system, utilizing Hypertext Preprocessor (PHP) as the primary programming language. The system aims to monitor environmental parameters such as temperature, humidity, and soil moisture in real-time, providing farmers with valuable insights into their crop conditions.\\
The system utilizes PHP for web development, temperature, humidity, and soil moisture sensors for data collection, and a microcontroller (Arduino) for interfacing sensors with the web application. Data is stored in a database and presented to users through a user-friendly graphical user interface (GUI), with control charts generated using Crystal Report.\\
The system allows real-time monitoring of environmental parameters, enabling farmers to promptly respond to changes. A user-friendly GUI is designed to facilitate ease of use for farmers, with categorized menus and monitoring buttons for each parameter. Data collected from sensors is stored in a database and presented to users in control charts, providing visual insights into crop conditions. The system can be accessed online, allowing farmers to monitor their crops remotely using various devices such as smartphones or tablets.\\
The system relies on internet connectivity for remote monitoring, which may pose challenges in areas with poor connectivity. Implementing the system requires the use of various hardware components such as sensors and microcontrollers, which may increase complexity and maintenance requirements.

\subsubsection{Design and Implementation of an IoT-Based System for Monitoring Nutrients and Irrigation in Agricultural Soil}
The work was conducted by Gomathi S, Raj Kumar V, and Vishnu Aakash R \cite{gomathi2022design} from the Department of Information Technology at Sri Sairam Engineering College in Chennai, Tamil Nadu, India.\\
The study focuses on the design and implementation of an IoT-based system for monitoring soil nutrients and irrigation in agricultural soil. It involves the development of hardware using Arduino and ESP8266 modules, along with various sensors such as soil moisture sensor, temperature sensor, and NPK sensor. The system aims to gather real-time data on crucial parameters including soil moisture, temperature, nitrogen, phosphorus, and potassium levels. Additionally, a smart irrigation system is presented to optimize water usage based on sensor data. The system involves the utilization of Arduino and ESP8266 modules interfaced with various sensors including capacitive soil moisture sensor, DS18B20 Waterproof Temperature Sensor, and soil NPK sensor. The system also incorporates a smart irrigation system for efficient water management.\\ The system provides real-time monitoring of crucial soil parameters, enabling farmers to make informed decisions. Utilization of IoT technology allows wireless monitoring and data collection, enhancing accessibility and convenience for farmers. Integration of a smart irrigation system helps optimize water usage, leading to improved crop yield and resource efficiency. The use of open-source hardware and sensors makes the system cost-effective and accessible to smaller farmers.\\
The study does not provide detailed information on the accuracy and reliability of the sensor measurements. The scalability and interoperability of the system with existing agricultural infrastructure are not extensively discussed. Limited discussion on potential challenges and limitations of implementing the IoT-based system in diverse agricultural settings.

\subsubsection{Design and Development of Internet of Things Based Smart Sensors for MOnitoring Agricultural lands}
This study was made by, Dhiya Sabu, Paramasivam Alagumariappan, Vijayalakshmi Sankaran and Pavan Sai Kiran Reddy Pittu \cite{sabu2023design} from Department of Electronics and Communication Engineering, Vel Tech Rangarajan Dr. Sagunthala R \& D Institute of Science and Technology, Chennai 600062, India.\\
In the study, the entire device components are arranged in a three-stack structure, in which the solar cell is placed at the top stack; the battery, microcontroller unit, DC-to-DC converter, and battery charger unit are mounted on the middle stack; and the sensors, namely the pH sensor and the soil moisture sensor, are arranged on the bottom stack. Also, the foam floats were attached to the bottom stack to make the device float if the land was filled with water.//
The proposed device was a made as a standalone device, in which the electrical energy is generated with the help of a solar cell. The output of the solar cell was 12 volts, and all the device components utilized in this work required a maximum of 5 volts. So, it was essential to step down the voltage level of the solar cell, and to achieve this operation, a DC-to-DC converter was used. The XY-3606-based DC-to-DC converter was used as a step-down converter, which reduced the voltage from 12 volts to 5 volts and supplied a maximum current of 5 A. iInce the solar irradiance may not be constant all the time, and, in turn, the electrical energy generated varies with time. To supply a constant source to all the device components, the battery was utilized. Further, a lithium polymer (LiPo) battery with 3.7 volts (1000 mAh) was used. Also, the LiPo battery was charged using the LiPo battery charger circuit.\\
In this work, two different sensors were used, namely the pH sensor and the soil moisture sensor. Further, the acidity and basicity of the soil were measured with the help of the pH sensor. Furthermore, the proposed pH sensor was capable of operating from 3.3 to 5.5 volts. Also, the same pH sensor module was utilized to measure the temperature of the soil. The soil moisture sensor module has a pair of electrodes and a comparator board that operates from 3.3 to 5 volts.\\
An ESP8266 microcontroller unit, otherwise known as the Node MCU, was utilized in this work to collect sensor data. Further, the Node MCU operates on a 5-volt power supply, which can be powered using a LiPo battery through the Pololu U3V70F5 board. The Node MCU has an in-built WiFi module that helps the user to feed their sensor data to the IoT cloud. The ESP8266 has one analog pin to which one sensor can be connected. So, the external analog-to-digital converter, namely the ADS1115, was connected to the utilized Node MCU. Further, the sensor data were converted into digital data, and these digital data were fed to the Node MCU via the inter-integrated circuit (I2C) protocol.\\
Two different IoT clouds, namely the ThingSpeak platform and the custom-designed IoT platform, were used in this work to log the parameters of the agricultural land. The ThingSpeak IoT cloud platform stores the data with respect to time. Also, the custom-designed IoT platform has various features to monitor the wetness, temperature, and pH of the soil.

\subsubsection{Design A Monitoring System for Temperature, Humidity and Soil pH in IOT-Based Onion Cultivation}
In an article by Salsabila Audyanisa and Abdi Darmawan \cite{audyanisa2024design} from Darmajaya Institute of Informatics and Business discuss a farming monitoring system tailored for onion cultivation.
The system in this study is divided into three parts, including an input system consisting of a soil pH sensor and a DHT 11 sensor. The microntroller used is a minimum board system NodeMCU ESP8266. The output system in the form of a water pump and Android application is used as a control and monitoring of sensor readings. The soil pH sensor serves as a measure of the value of the soil in onion plants. The DHT 11 sensor is used as a temperature and humidity reader. The microcontroller system in this design uses the NodeMCU minimum system board ESP8266. The relay functions as the ON and OFF water pump applied to the system. The application functions as a minitoring of soil pH sensor readings, DHT 11 sensors and water pump controls.

\subsubsection{Design and Development of Soil Monitoring System for Precision Farming On Small-Scale Outdoor Farm}
This paper was authored by Dezdyta Poszarevac Saputra, M.B. Nugraha, Marojahan Tampubolon, Kahfi Sabillah Arhan \cite{10435334} from Electrical Engineering, Universitas Multimedia Nusantara, Tangerang, Indonesia.
Their research consisted of a subsystem for monitoring plant conditions and nutrition and a user interface (UI) subsystem. This product has a subsystem for monitoring soil conditions and plant nutrition with parameters measured, namely ambient temperature, soil moisture, air humidity, light, and CO2 levels, so that plants’ water and nutrient needs can be met to fulfil the physical demands of vegetable crops. It can be fulfilled according to the parameters set and measured by the sensor. The user interface subsystem functions to display data from sensor parameters of ambient temperature, soil humidity, air humidity, light intensity, CO2 levels, and NPK through the Node-RED server.\\
The system developed in this research offers a valuable advantage by effectively addressing user requirements in an automated or smart farming industry scene. When users seek information regarding the physical condition of their plants, they can utilize various sensors to obtain precise measurements. The DHT-11 sensor provides data of environmental temperature, the YL-69 sensor offers soil moisture measurements in percentage units, while air humidity is recorded by the DHT-11 sensor. CO2 levels are monitored through the MQ-135 sensor. Sunlight intensity readings, gathered via the BH1750 sensor, are expressed in lux units, and plant’s nutrition in NPK values are assessed using custom sensors from Doctor Plant products(). These sensors collectively yield real-time measurement data, which is conveniently presented to users through the user interface subsystem. The user interface effectively showcases information related to ambient temperature, soil humidity, air humidity, light intensity, CO2 levels, and NPK values, seamlessly delivered via the Node-RED server.

\subsubsection{IOT Based Real-Time Monitoring System for Precision Agriculture}
This paper is authored by Surbhi Vijh, Arpita, Jyoti Prakash Bora, Prateek Kumar Gupta from School of Engineering and Technology, Sharda University, Greater Noida, India and Sumit Kumar from ASET, Amity University, Noida, Uttar Pradesh \cite{10463399}\\
The project uses a DHTll Humidity Temperature Sensor to track humidity as well as air temperature, and a capacitive soil moisture sensor to measure soil moisture levels. A 5V power relay-controlled water pump automates the irrigation process. The sensor makes sure that the soil stays sufficiently moist by activating a water pump to start irrigation whenever the moisture content of the soil drops below a predetermined level. An online Thing speak Server facilitates the system's real-time monitoring feature  with the help of this server, farmers can remotely monitor and control their agricultural operations from any location in the globe. They can also use an easy-to-use online interface to make informed decisions about irrigation and to closely monitor soil conditions.

\subsubsection{Enabling Smart Agriculture: An IoT-Based Framework for Real-Time Monitoring and Analysis of Agricultural Data}
This paper was authored by Faruk Enes Oguz, Mahmut Nedim Ekersular, Kubilay Muhammed Sunnetci and Ahmet Alkan. \cite{oguz2024enabling}
In this study, an IoT-based framework is proposed for agricultural data monitoring. Light, temperature–pressure, smoke, humidity, and soil dryness values can be measured from GY-30, BME280, MQ-2, DHT11, and YL-69, respectively. An ESP-32S development board is used to collect data from sensors, and this board is coded using Arduino IDE. Subsequently, using ESP-32S, it is sent to the ThingSpeak cloud service provided by MATLAB via a Wi-Fi connection. Thus, these data can be easily transferred to MATLAB. We create a user-friendly Graphical User Interface application so that the data can be monitored and analyzed in MATLAB as well as ThingSpeak. This application allows users to monitor the data flow in real time and can easily provide the requested values such as maximum, minimum, mean, standard deviation, and current with the help of a button. In addition, the proposed system sends an e-mail to the user when soil dryness and smoke values exceed a certain threshold value.
\subsection{Summary}
\newpage

\section{Methodology}
\subsection{Overall System Architecture}
\subsection{Hardware Architecture}
\subsection{Software Architecture}
\newpage

\section{References}
\bibliographystyle{plain}
\bibliography{references}

\end{document}
