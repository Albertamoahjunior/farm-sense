\documentclass[12pt, a4paper]{article}
\usepackage{graphicx}
\graphicspath{{./images}}

\title{ALLOCATION OF SENSORS TO MONITOR VITAL SOIL AND ATMOSPHERIC FACTORS THAT AFFECT PLANT GROWTH
\includegraphics[scale=0.6]{knust}}

\begin{document}

\maketitle
\section{Introduction}
\subsection{Background of Study}
Farming is a very important aspect of humanity. Humans cannot live without food and food cannot be without farms, hence the neccessity of farms. Farmers do a great service to man. As the SDG goal number goes: \textbf{Zero Hunger}; End hunger, achieve food security and improved nutrition and promote sustainable agriculture. To achieve this we need to take care of agriculture and improve on its productivity. Agricultural productivity is key to eradicating hunger and achieving food security. As stated in the SDG goal, "promote sustainable agriculture" sustainable agriculture is a much needed achievement. Agricultural productivity can be improved by lay down and implementing measures that maximise crop yield. There are various many factors that affect plant growth and therefore their yield. Plants need certain essentials to grow well and these essentials if provided accurately can increase their growth and therefore increase productivity.
\\
Several factors affect crop health and contribute in the growth of plants soil factors, atmospheric factors and pests\\Soil factors include soil temperature, the soil pH, supply of soil nutrients, soil moisture etc. Atmospheric factors also include atmospheric temperature, humidity, precipitation or rain etc. Soil temperature directly affects plant growth. Most soil organisms function best at an optimum soil temperature. Soil temperature impacts the rate of nitrifi- cation. It also influences soil moisture content, aeration and availability of plant nutrients. Soil pH will influence both the availability of soil nutrients to plants and how the nutrients react with each other. For example: At a low pH, many elements become less available to plants, while others such as iron, aluminum and manganese become toxic to plants. Soil moisture is a measure of soil health, the water content present in a certain area of the ground. All plants need to be in a specific soil moisture range — the majority of plants thrive in soil with a moisture level that ranges between 20 and 60 percent. This is important because if you're trying to grow certain plants, you not only have to make sure the soil is fertile enough to support growth, you have to be able to keep soil moisture in a certain range. It's an issue for industrial growers, but also private gardeners and anyone trying to grow their own vegetables. This is because the water content in soil is a solvent, meaning that it breaks down the nutrients and minerals that plants need from the dirt, allowing them to absorb these helpful particles into their systems. As atmospheric temperature increases (up to a point), photosynthesis, transpiration and respiration increase. When combined with day length, temperature also affects the change from vegetative (leafy) to reproductive (flowering) growth. Too much rain during germination may saturate soils, resulting in poor germination and reduced stands. However, too little rain during germination may reduce germination and leave plants ill-prepared for future growth and development challenges. When relative humidity levels are too high or there is a lack of air circulation, a plant cannot make water evaporate (part of the transpiration process) or draw nutrients from the soil. When this occurs for a prolonged period, a plant eventually rots.
\subsection{Problem Statement}
Agricultural farms are important to humans. They produce the food we eat. The food we eat are the yields of the crops that are cultivated.\\ \\
Farmers frequently face challenges in providing the necessary farm essentials to support ideal plant growth, primarily due to inadequate and inaccurate farm condition data. The problem is that farmers have been providing farm essentials to crops based on guess work. They are not able to accurately tell the needs of the farms.
\\
Current farm management practices are often reactive and data-deficient, leading to inefficiencies and unsustainable resource use across various aspects of agricultural production Farmers struggle to optimize irrigation, fertilization, and pest control due to limited real-time data on soil health, weather conditions, and crop development, resulting in potential yield losses, water waste, and environmental damage from excess nutrients.
\\
Therefore, there is a critical need for a farm monitoring system that:
\begin{itemize}
 \item Provides real-time, comprehensive data on environmental conditions, crop health.
 \item Offers user-friendly interfaces and actionable insights for informed decision-making.
 \item Is cost-effective, scalable, and accessible to farmers of all sizes.
 \item Contributes to sustainable agriculture practices by optimizing resources
\end{itemize}
By addressing these challenges a well-designed system that can monitor vital soil and atmospheric factors that affect farm plant growth can empower farmers to achieve: increased productivity and profitability, enhanced decision-making and long-term sustainability. \\ \\



\newpage
\subsection{Objectives of the Project}
\subsubsection{General Objectives}
The project's objective is to develop a streamlined approach for deploying sensors to monitor critical soil and atmospheric conditions essential for plant growth. This involves identifying the optimal quantity, types, and placements of sensors to collect data on variables such as moisture levels, nutrient content, temperature, humidity, and atmospheric gases. Ultimately, the aim is to equip farmers with timely information to refine their farming techniques, resulting in higher crop yields and quality, decreased resource consumption, and a reduced environmental impact.
\subsubsection{Specific Objectives}
This project aims to achieve several key objectives:\\
\textbf{Increased efficiency and productivity:}
\begin{itemize}
 \item Optimize resource use (water, fertilizer, energy) based on real-time data.
 \item Automate tasks like irrigation and climate control to save time and labor.
 \item Improve yield and crop quality through early detection of problems.
\end{itemize}
\textbf{Reduced environmental impact:}
\begin{itemize}
 \item Minimize water usage and fertilizer runoff through precise application.
 \item Track and manage greenhouse gas emissions from livestock and agricultural activities.
 \item Promote sustainable farming practices for long-term environmental health.
\end{itemize}
\textbf{Improved decision-making:}
\begin{itemize}
 \item Gain data-driven insights to optimize planting, harvesting, and pest control strategies.
 \item Forecast potential risks and proactively prepare for weather events or disease outbreaks.
 \item Facilitate informed decision-making for improved farm management and profitability.
\end{itemize}

\newpage
\subsection{Scope of the project}

\newpage
\subsection{Significance of Study}
Allocation of sensors to monitor vital soil and atmospheric factors is very essential. The can help improve on our agricultural prowess and productivity. 
\\ \\
\textbf{Optimization of Resources:} The system can help farmers optimize the use of resources such as water, fertilizers, and pesticides, reducing waste and increasing efficiency.
\\ \\
\textbf{Early Detection of Issues:} This system can detect issues such as pest infestations, disease outbreaks, or nutrient deficiencies early, allowing farmers to take timely action to mitigate losses.
\\ \\
\textbf{Data-Driven Decision Making:} By collecting and analyzing data on crop health, weather patterns, soil moisture, and other factors, farmers can make informed decisions to improve yields and profitability.
\\ \\
\textbf{Remote Monitoring:} With this system, farmers can remotely monitor their fields allowing them to keep an eye on their operations even when they are not physically present.
\\ \\
\textbf{Environmental Sustainability:} By monitoring factors like soil health and water usage, farmers can adopt more sustainable practices that minimize environmental impact and conserve natural resources.
\\ \\
\textbf{Increase in Productivity:} By providing real-time insights into crop conditions, monitoring systems can help farmers increase productivity and ultimately profitability.


Overall, farm monitoring systems enable farmers to make more informed decisions, increase efficiency, and ultimately improve the sustainability and profitability of their operations.

\newpage
\subsection{Organisation of Study}
\begin{enumerate}
\item \textbf{Introduction}
	\begin{itemize}
	\item[--] Aim and objectives of the project.
	\item[--] Background information on how certain soil and atmospheric factors affect plant    growth.
	\end{itemize}
\item \textbf{Farm monitoring systems}
\begin{itemize}
    \item[--] Types of farm monitoring systems implemented.
	\item[--] Benefits and limitations of existing technology.
\end{itemize}
\item \textbf{Design of System}
\begin{itemize}
	\item[--] Design of system architecture.
    \item[--] Physical design consideration for the system.
    \item[--] Components of the smart monitoring system.
\end{itemize}
\item \textbf{Programming of the System}
\begin{itemize}
	\item[--] Appropriate programming laguages that can be used for programming the system and controling the various components.
	\item[--] Algorithims that would be used to gather and analyze information.
\end{itemize}
\item \textbf{Prototype development}
\begin{itemize}
\item[--] Building a prototype of the monitoring system
\item[--] Testing the prototype for efficiency and functionality
\item[--] Iterative improvement of the prototype system
\end{itemize}
\item \textbf{Conclusion}
\begin{itemize}
	\item[--] Summary of the study.
	\item[--] Future scope of the project.
\end{itemize}
\end{enumerate}

\end{document}